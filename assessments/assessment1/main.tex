\documentclass[12pt,letterpaper]{article}
\usepackage{fullpage}
\usepackage[top=2cm, bottom=4.5cm, left=2.5cm, right=2.5cm]{geometry}
\usepackage{amsmath,amsthm,amsfonts,amssymb,amscd}
\usepackage{lastpage}
\usepackage{enumerate}
\usepackage{fancyhdr}
\usepackage{mathrsfs}
\usepackage{xcolor}
\usepackage{graphicx}
\usepackage{listings}
\usepackage{hyperref}

% Edit these as appropriate
\newcommand\course{Algorithms Lab HS2021}
\newcommand\hwnumber{1}
\newcommand\NetIDa{Andrea}
\newcommand\NetIDb{Pinto}

\pagestyle{fancyplain}
\headheight 35pt
\lhead{\NetIDa}
\lhead{\NetIDa\\\NetIDb}
\chead{\textbf{\Large Assessment \hwnumber}}
\rhead{\course \\ \today}
\lfoot{}
\cfoot{}
\rfoot{\small\thepage}
\headsep 2.5em

\begin{document}
\small
\subsection*{Code submission: 100\%}
\begin{verbatim}
#include <iostream>
#include <vector>

using namespace std;
typedef vector<int> vint;

int n, m, k;

int profit(vint &coins, vector<vint> &memo, int i, int j) {
  if (j < i) return 0;
  if (memo[i][j] != -1) return memo[i][j];
  
  int p1 = profit(coins, memo, i + 1, j);
  int p2 = profit(coins, memo, i, j - 1);
  bool myturn = (n - (j - i + 1)) % m == k;
  
  if (myturn) return memo[i][j] = max(coins[i] + p1, coins[j] + p2);
  else return memo[i][j] = min(p1, p2);
}

void solve() {
  cin >> n >> m >> k;
  vint coins(n);
  vector<vint> memo(n, vint(n, -1));
  for (int i = 0; i < n; i++) cin >> coins[i];
  cout << profit(coins, memo, 0, n - 1) << endl;
}

int main() {
  ios_base::sync_with_stdio(false);
  int t; cin >> t;
  for (int i = 0; i < t; i++) solve();
}
\end{verbatim}
\end{document}